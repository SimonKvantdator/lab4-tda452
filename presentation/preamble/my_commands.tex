%! TEX root = /home/simon/Documents/masterarbete/presentation/presentation.tex

% Vectors are upright boldface. I think this definition is better than the physics package's \vectorbold.
\let\Vec\undefined % We use \vec w/ lowercase v
\renewcommand*{\vec}[1]{{\boldsymbol{\mathrm{#1}}}}

% Bar, tilde, and hat that scales with what is under them. Basically I just want these to have consistent names
\let\mathbar\overline
\let\mathtilde\widetilde
\let\mathhat\widehat

% Redefine \exp
% Errors occur if this definition is made before some of the packages are loaded
\let\oldexp\exp
\newcommand*{\Exp}[1]{\oldexp{#1}}
\renewcommand{\exp}[1]{\mathrm{e}^{#1}}

% Main number systems
\newcommand{\naturals}{\mathbb{N}}
\newcommand{\integers}{\mathbb{Z}}
\newcommand{\rationals}{\mathbb{Q}}
\newcommand{\reals}{\mathbb{R}}
\newcommand{\complexnumbers}{\mathbb{C}}

% Some of my own shorthands for correct spacing in math environments
\def\divides{\mid} % Proper spacing of vertical bar in division x|y
\def\from{\colon} % Proper spacing of colon in functions f:A→ B
\newlength\mylen % Isomorphic \mapsto
\settowidth\mylen{$\longleftrightarrow$}
\newcommand{\mapsbetween}{\longleftrightarrow\kern - 0.5\mylen\vline height 1.2ex depth -0.0pt\kern0.5\mylen}
\newcommand{\suchthat}{\qq{s.th.}}
\def\definedas{\coloneqq}
\def\defines{\eqqcolon}

\newcommand*{\transpose}[1]{{#1}^{\!\mathsf{T}}}
\renewcommand*{\complement}[1]{{#1}^{\mathsf{C}}}
% \newcommand{\conjugate}[1]{\stackrel{\rule{0.4em}{0.3pt}}{#1}}
% \newcommand{\conjugate}[1]{\mathbar{#1}}
% \newcommand*{\conjugate}[1]{{#1}^*}
\newcommand{\conjugate}[1]{\mkern 1.0mu\overline{\mkern-1.0mu#1\mkern-1.0mu}\mkern 1.0mu}
\newcommand*{\hermitianconjugate}[1]{{#1}^\dag}
\newcommand*{\inverse}[1]{{#1}^{-1}}
\newcommand*{\dual}[1]{{#1}^{*}} % dual vector space

\newcommand*{\closure}[1]{\mathbar{#1}} % Closure of a set
\def\union{\cup}
\newcommand{\Union}{\bigcup\limits}
\def\intersection{\cap}
\newcommand{\Intersection}{\bigcap\limits}

% Lie-groups & algebras, i.g. SU(n)
\newcommand*{\algebra}[2]{{\mathfrak{\MakeLowercase{#1}}}{\left(#2\right)}}
\newcommand*{\group}[2]{{\mathrm{\MakeUppercase{#1}}}{\left(#2\right)}}

% Fundamental operators
\newcommand{\fundamentaldivergence}{\mathscr{D}}
\newcommand{\fundamentalcurl}{\mathscr{C}}
\newcommand{\fundamentaltwistor}{\mathscr{T}}

\newcommand{\sDiv}{\fundamentaldivergence}
\newcommand{\sTwist}{\fundamentaltwistor}
\newcommand{\sCurl}{\fundamentalcurl}
\newcommand{\sCurlDagger}{\fundamentalcurl^\dagger}

% Symmetric multiplication
\newcommand{\SymMult}[2]{\overset{#1, #2}{\odot}}

% To force some equations to be numbered
\newcommand\hiddenref[1]{\sbox0{\cref{#1}}}

% SymH
\DeclareMathOperator*{\Sym}{Sym}
