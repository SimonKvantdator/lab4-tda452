%! TEX root = /home/simon/Documents/masterarbete/presentation/presentation.tex

% ### From template ###

\mode<presentation>
\usepackage{pgfpages}


%%%%%%%%%%% LANGUAGE %%%%%%%%%%%

% For correct hyphenation in swedish
\usepackage{fontenc}

% For interpreting non-ASCII characters
\usepackage[utf8]{inputenc}

% International language support
% Fetches language from documentclass options. Most other packages do this as well
\usepackage{babel}


%%%%%%%%%%% FORMAL STUFF %%%%%%%%%%%

% Dates & time
\usepackage[yyyymmdd]{datetime} % Useful when referencing websites
\renewcommand{\dateseparator}{-} % ISO 8601 date format


% Citing & bibliography
\usepackage{csquotes} % For \enquote command for proper quotation marks, also biblatex recommends this
\usepackage[numbers]{natbib}


%%%%%%%%%%% GRAPHICS %%%%%%%%%%%

% \usepackage{graphics, color}
% \usepackage[table]{xcolor}

% Figures
% \usepackage{epsfig} % Solves some problems in \includegraphics{<.eps-file>}
% \usepackage{graphicx} % More options for \includegraphics
% \usepackage{float} % Figure placement
% \usepackage[labelfont=bf, textfont=normal,justification=justified,singlelinecheck=false]{caption} % More options for \caption
% \usepackage{subcaption} % Subfigures

% Tikz
% \usepackage{tikz}
% \usepackage{pgf, pgfplots} % Pgfplot
% \pgfplotsset{compat=1.15}

% För alduslöv
\usepackage{pifont}


%%%%%%%%%%% PHYSICS %%%%%%%%%%%

% SI units
\usepackage{siunitx}
\DeclareSIUnit\clight{\text{$c$}} % redefine from c_0 to c
\DeclareSIUnit\byte{B}

% Physics macros
\usepackage{physics} % Defines lots of nice commands like \derivative, \norm, \evaluated, etc. It is recommended to use these as much as possible for nice spacing and readable LaTeX code.
\usepackage{braket} % Defines \bra, \ket, \braket, and \set
\usepackage{tensor} % Covariant index notation
\usepackage[]{slashed} 


%%%%%%%%%%% CODING %%%%%%%%%%%

% For nice code insertions
\usepackage{listings}
\lstloadlanguages{Mathematica}
\definecolor{codegreen}{rgb}{0,0.6,0}
\definecolor{codegray}{rgb}{0.5,0.5,0.5}
\definecolor{codepurple}{rgb}{0.58,0,0.82}
\definecolor{backcolour}{rgb}{0.95,0.95,0.92}
\lstdefinestyle{mystyle}{
    backgroundcolor=\color{backcolour},   
    commentstyle=\color{codegreen},
    keywordstyle=\color{magenta},
    numberstyle=\tiny\color{codegray},
    stringstyle=\color{codepurple},
    basicstyle=\ttfamily\footnotesize,
    breakatwhitespace=false,         
    breaklines=true,                 
    captionpos=b,                    
    keepspaces=true,                 
    numbers=left,                    
    numbersep=5pt,                  
    showspaces=false,                
    showstringspaces=false,
    showtabs=false,
    tabsize=4
}
\lstset{style=mystyle}


%%%%%%%%%%% MATHEMATICS %%%%%%%%%%%

% AMS packages
\usepackage{amsmath}
\usepackage{amsfonts}
\usepackage{amsthm}
\usepackage{amssymb}
\usepackage{mathrsfs}

% Better version of the \not command
\usepackage{cancel}

%%%%%%%%%%% MISCELLANEOUS %%%%%%%%%%%

% Clickable links and refs
% \usepackage{hyperref}								
% \hypersetup{final, colorlinks, citecolor=black, filecolor=black, linkcolor=black, urlcolor=black}

% Cleverref automatically detects if you are referencing a figure, table, or equation etc
% Cleverref has to be loaded last I think, after babel and hyperref etc
% \usepackage[noabbrev, nameinlink]{cleveref}
% \crefname{equation}{}{}
% \iflanguage{swedish}{ % Tell cleverref to use Oxford comma
% 	\newcommand{\creflastconjunction}{, och\nobreakspace}
% }{}
% \iflanguage{english}{
% 	\newcommand{\creflastconjunction}{, and\nobreakspace}
% }{}

% Theorem and proof environments
% Has to be put after loading cleveref
% \iflanguage{swedish}{
%     \newtheorem{theorem}{Sats}
%     \newtheorem*{theorem*}{Sats}
%     \newtheorem{proposition}{Proposition}
%     \newtheorem*{proposition*}{Proposition}
%     \newtheorem{corollary}{Följdsats}[theorem]
%     \newtheorem{corollary*}{Följdsats}
%     \newtheorem{lemma}{Lemma}
%     \newtheorem*{lemma*}{Lemma}
%     \newtheorem{remark}{Kommentar}
%     \newtheorem*{remark*}{Kommentar}
%     \theoremstyle{definition}
%     \newtheorem{definition}{Definition}
%     \newtheorem*{definition*}{Definition}
% }{}
% \iflanguage{english}{
%     \newtheorem{theorem}{Theorem}
%     \newtheorem*{theorem*}{Theorem}
%     \newtheorem{proposition}[theorem]{Proposition}
%     \newtheorem*{proposition*}{Proposition}
%     \newtheorem{corollary}{Corollary}[theorem]
%     \newtheorem{corollary*}{Corollary}
%     \newtheorem{lemma}[theorem]{Lemma}
%     \newtheorem*{lemma*}{Lemma}
%     \newtheorem{remark}[theorem]{Remark}
%     \newtheorem*{remark*}{Remark}
%     \theoremstyle{definition}
%     \newtheorem{definition}[theorem]{Definition}
%     \newtheorem*{definition*}{Definition}
% }{}

% Intervals on the real line
\let\interval\undefined % To avoid name conflict with etextools
\usepackage{interval}
\intervalconfig{soft open fences}
